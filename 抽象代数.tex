\documentclass[12pt,a4paper,UTF8]{ctexbook}
\usepackage[top=2cm, bottom=2cm, left=2cm, right=2cm]{geometry} % 页边距
\usepackage{amsmath,amssymb}    % 数学公式
\usepackage{graphicx}           % 图片插入
\usepackage{booktabs}           % 专业表格
\usepackage{multirow}            % 表格跨行
\usepackage{amsthm}
\usepackage[colorlinks=true]{hyperref} % 超链接
\usepackage{tikz}
\usepackage{pgfplots}

%自定义宏
%公式命令宏
\theoremstyle{plain}
\newtheorem{theorem}{\indent Theorem}[section]
\newtheorem{lemma}[theorem]{\indent Lemma}
\newtheorem{proposition}[theorem]{\indent Proposition}
\newtheorem{corollary}[theorem]{\indent 推论}
\newtheorem{definition}{\indent Definition}[section]
\newtheorem{example}{\indent Example}[section]
\newtheorem{remark}{\indent Remark}[section]
\newenvironment{solution}{\begin{proof}[\indent\bf Solution]}{\end{proof}}
\newtheorem{axiom}{\indent Axiom}[section]
\renewcommand{\proofname}{\indent\bf Proof}
% 快速输入数学符号
\newcommand{\R}{\mathbb{R}}    % 实数集
\newcommand{\C}{\mathbb{C}}    % 复数集
\newcommand{\Z}{\mathbb{Z}}    % 整数集
\newcommand{\Q}{\mathbb{Q}}    % 有理数集
\newcommand{\N}{\mathbb{N}}    % 自然数集

% 微分符号 (直立体)
\newcommand{\diff}{\mathop{}\!\mathrm{d}} % 用法: \int f(x) \diff x

% 概率期望/方差
\newcommand{\E}{\mathbb{E}}    % 期望
\newcommand{\Var}{\operatorname{Var}}    % 方差

% 自动调整括号大小
\newcommand{\paren}[1]{\left(#1\right)}   % 圆括号
\newcommand{\bracket}[1]{\left[#1\right]} % 方括号
\newcommand{\braces}[1]{\left\{#1\right\}} % 花括号

% 向量/矩阵(粗体)
\newcommand{\vb}[1]{\mathbf{#1}}          % 向量 \vb{v}
\newcommand{\mat}[1]{\mathbf{#1}}         % 矩阵 \mat{A}
\newcommand{\tran}{^{\mathsf{T}}}         % 转置 A\tran

\title{\bf{Abstract Algebra}}        
\author{陈彦桥\\
        南方科技大学}
\begin{document}

\maketitle
\tableofcontents

\chapter{群、环、体、域}
\section{群}
\subsection{群的定义}
\begin{definition}[群]
    若一个集合$G$定义了一个二元运算$\circ$,满足
    \begin{itemize}
        \item 运算封闭性:若$a,b \in G$,则$a\circ b\in G$;
        \item 结合律:$(a\circ b)\circ c = a\circ (b\circ c)$;
        \item 单位元:存在$e$使得对于$a\in G$,有$e\circ a = a\circ e = a$;
        \item 逆元:存在$b:=a^{-1}$使得对于$a\in G$,有$a\circ a^{-1} = e$;
    \end{itemize}
    则$G$是一个群。
\end{definition}
通常,我们会省略写中间的运算符号,即$a\circ b = ab$。
\begin{remark}
    进一步,若满足交换律$a\circ b = b\circ a$,则称这个群是阿贝尔群(Abel Group)。
\end{remark}
对于代数结构没有特别好的一些结构,我们同样予以其命名:
\begin{definition}[半群]
    若一个集合$G$定义了一个二元运算$\circ$,满足
    \begin{itemize}
        \item 运算封闭性:若$a,b \in G$,则$a\circ b\in G$;
        \item 结合律:$(a\circ b)\circ c = a\circ (b\circ c)$;
    \end{itemize}
    则$G$是一个半群(Semigroup)。
\end{definition}
\begin{remark}
    进一步,若半群具有单位元,那么称之为幺半群(Monoid)
\end{remark}
\begin{definition}[阶]
    群中元素的数量被称为阶。
\end{definition}

\subsection{群的基本性质}
\begin{theorem}[基本性质]
    以下的几条定理是不言而喻的:
    \begin{itemize}
        \item 群的单位元唯一。
        \item 群的任意元素的逆元唯一。
        \item 群具有消去律。(这由逆元的性质保证)
    \end{itemize}
\end{theorem}
\begin{definition}[子群]
    若集合H满足以下性质:
    \begin{itemize}
        \item 群$G$元素构成的集合的子集为$H$;
        \item $H$满足$G$中运算的所有性质;
    \end{itemize}
    则$H$是$G$的子群,记为$H\leq G$。
\end{definition}
\begin{theorem}[子群的判定定理]
    下述三个条件等价:
    \begin{itemize}
        \item $H\leq G$;
        \item $\forall a,b\in G$,$ab\in H$且$a^{-1}\in H$;(平凡验证封闭性与逆元)
        \item $\forall a,b\in G$,$ab^{-1}\in H$(或者$a^{-1}b\in H$);(实用的推论)
    \end{itemize}
\end{theorem}
\begin{definition}[群的积]
    定义: 
    \begin{align*} 
        HK = \{hk|h\in H, k\in K\}
    \end{align*} 
\end{definition}
将判定定理改述为上述语言得到:
\begin{theorem}[子群的判定定理(改述版本)]
    对于$H \subseteq G, H\ne \emptyset$,下述三个条件等价:
    \begin{itemize}
        \item $H\leq G$;
        \item $H^2 \subseteq H$且$H^{-1}\subseteq H$;
        \item $HH^{-1}\subseteq H$(或者$H^{-1}H\subseteq H$);
    \end{itemize}
\end{theorem}
\begin{definition}[平凡子群、真子群、生成群、生成系]
    作出如下定义:
    \begin{itemize}
        \item 显然$G$的子群可以为其自身以及\{e\},\{$e$\}被称为$G$的平凡子群。
        \item 若$H\leq G$,且$H\leq G$,则$H$为$G$的真子群。
        \item 若$M\subseteq G$,$\forall (M\subseteq H_i,H_i\leq G)$,$\bigcap_{i=1}^n H_i$被称为“$M$生成的子群”,记为$\langle M\rangle $。
        \item 若$\langle M\rangle =G$,称$M$为G的一个生成系。
        \item 若$|M|=1$且$M$为$G$的一个生成系,则$G$为循环群。
        \item 由有限个元素生成的群为有限生成群。
        \item 对于任意元素$a$,称$\langle a\rangle$的阶为元素$a$的阶,记为$o(a)$。
        \item 群中所有元素阶的最小公倍数为群的方次数,记为$\exp(G)$。
    \end{itemize}
\end{definition}
\begin{remark}
    实际上,“生成”可以理解为$M$的若干元素包含在若干个封闭的子群中,而这些子群的交恰好就是$M$中元素“能够”通过子群中运算得到的
    “被生成的元素”。对于这些子群中的其它元素,需要不属于$M$中的元素通过运算生成,因此也就自然不属于$M$的“生成”结果。
\end{remark}
\subsection{陪集(Coset)}
接下来,我们进入群理论中第一个非常重要的部分:陪集。
\begin{definition}[右(左)陪关系]
    设$H\leq G$,定义等价关系$a\overset{l}{\sim}b$为
    \begin{align*} 
        \exists h\in H, a=bh.
    \end{align*}
    同样地可以定义$a\overset{r}{\sim}b$
    \begin{align*} 
        \exists h\in H, a=hb.
    \end{align*}
    注意到$a\overset{l}{\sim}b$意味着$a\in bH$,则$bH$为一个等价类。同理可以得到另一个等价类$Hb$。
\end{definition}
\begin{remark}
    事实上,上述的等价关系阐明的是元素之间的转换关系。若一个元素可以通过一次左乘或者右乘某个子群中
    的元素而变成群中的另一个。由于a,b是任意的,我们很容易得出一个直观的结论:$H$相当于一个待选的操作集合,
    而整个群中的某个子集中的元素可以通过这些操作相互转化。一个不太恰当的比喻是,某些场所中的陪酒小姐,她可以
    只服务某一个人群,我们总能够在这一群小姐中找到一位,服务了$a$以后就去服务$b$。这样,我们就建立了一个人群
    的关系网络。而我们将要介绍的陪集,就是$H$要去陪的那些人。
\end{remark}
于是我们可以通过上述等价关系定义陪集:
\begin{definition}[陪集]
    若$H\leq G$,$a\in G$,形如$aH$(类似地,Ha)的集合被称为$H$的一个左(右)陪集。
\end{definition}
\begin{remark}
    我们之前的remark中的比喻,可以用来解释陪集的定义:若b在陪集中,那么他和a有过相同的陪酒小姐。这样,他们之间就
    存在一个中间人的关系。
\end{remark}
\begin{remark}
    形式上,一个关键洞察是,陪集是对于群结构的一个划分,这个划分由确定的集合决定。群的元素被划分为若干个等价类,从而方便我们
    进行研究。
\end{remark}
左陪集是等价类,因此群可以被分解为左陪集的无交并。
\begin{theorem}
    \begin{align*}
    G = \dot{\bigcup}_{aH} aH
    \end{align*}
\end{theorem}
\begin{remark}
    这个定理表明,这一群陪酒小姐服务的对象是分成了几个群体的,有一些陪酒小姐服务高端人士,有一些陪酒小姐服务中产阶级,
    有一些陪酒小姐则服务比较寒酸的人士。他们找的小姐的服务方针是不太一样的。这样,我们就可以通过小姐的档次分出群体中
    所有人的档次了。
\end{remark}
\begin{definition}[指数]
    $H$的左陪集的个数被称为$H$在$G$中的指数,记为$|G:H|$。
\end{definition}
注意到,$H$与其陪集存在双射,也就是
\begin{align*}
    \varphi:H\to aH 
\end{align*}
是双射。$H$一旦确定,陪集就是确定的。
\begin{theorem}[拉格朗日定理]
    $G$为有限群,$H\leq G$,则
    \begin{align*} 
        |G| = |G:H||H|
    \end{align*} 
\end{theorem}
这个定理在无交并和双射的前置结论下是显然的。
\begin{theorem}
    有限群的任意元素的阶整除群的阶,即$a^{|G|} = e$。
\end{theorem}
这里,后面的结论在前面那个条件的结论下显然。但是前面那个条件是为什么呢?根据拉格朗日定理,取$H = \langle a\rangle$即可。
\subsection{正规子群}
在线性代数中,商空间对于我们解决子空间相关的问题至关重要。显然,商空间$V/W = \{\overline\alpha|\overline\alpha = \alpha + W, \alpha\in V\}$
是$W$的一个关于加法的陪集。其中商空间中的运算被定义为$\overline{\alpha} + \overline{\beta} = \overline{\alpha + \beta}$,也就是$\alpha + W + \beta + W = (\alpha + \beta) + W$。
我们想要将这一思想推广到群中,但是,显然我们必须要解决左陪集和右陪集不同的问题。
\begin{theorem}
    $H\leq G$,任意两个左陪集相乘仍然为左陪集的充要条件是左右陪集相等。
\end{theorem}
进一步,为了定义商群,我们定义正规子群
\begin{definition}[正规子群]
    $H\leq G$,若$\forall a$,$aH = Ha$,则$H$为$G$的正规子群,记为$H\trianglelefteq G$。
\end{definition}
\begin{remark}
    我们继续用不正经的比喻来解释这一定义。首先,我们定义服务1和服务2(对应左陪和右陪),这样,某一个客人点了不同的服务,小姐的下一个客人
    是可能不一样的。但是,若这是一个“正规”的场所,那么服务不应该有“不正经”的内容,应该表里如一,因此不论客人点哪个服务,下一位客人都是一样
    的。
\end{remark}
\begin{theorem}[正规子群判定定理]
    以下三个条件等价:
    \begin{itemize}
        \item $H\trianglelefteq G$;
        \item $\forall a\in G,a^{-1}H a = H$;
        \item $\forall a\in G,a^{-1}h a \in H$;
    \end{itemize}
    其中最后一个条件比较实用。
\end{theorem}
进一步我们定义商群:
\begin{definition}[商群]
    若$H\trianglelefteq G$,则H的陪集在乘法下构成群,称其为G关于H的商群,记为$G/H$。
\end{definition}
\begin{remark}
很简单,$H$是小姐,拿着一批小姐名单去找人,找到的可不就是客户嘛。所以商群就是小姐的客户。
\end{remark}
\begin{remark}
对于陪集,我们可以有一个类比认知:我们小学就学过的除法中,$a/c = b$,其中b称为商,我们发现,陪集被定义为$a = bH$,
实际上就是说,$H$就是那个除数,它将群元素$a$的其它“结构”除掉,保持最本质的性质。这样,我们就不难理解为什么$H$是商群的
单位元了:在同余类中,无论加上模数多少遍,元素仍然保持不变。这恰好就是单位元的性质。我们将这一不变性推广到更广泛的区域上去,
就是我们的陪集。
\end{remark}

\section{同态与同构}
\subsection{同态与同构的定义与性质}
\begin{definition}[同态]
    若映射
    \begin{align*} 
    \varphi:G_1\to G_2
    \end{align*}
    保持运算,即$\varphi(ab)=\varphi(a)\varphi(b)$,则$\varphi$是一个同态。若$\varphi$为单(满)射,则为单(满)同态。若为双射,那么称之为同构,记为$G_1 \cong G_2$。
\end{definition}
\begin{remark}
同态表明,我可以将旧的经验转移到新的情境中,原有的群中的运算,可以分别将运算对象映射到像空间中再利用那里的运算得出结果。相当于运送建材,先在工厂用
电钻拆开,然后拿到工地也能拿螺丝刀拧上。这就是同态,一个样。
\end{remark}
\begin{definition}[自同态]
    若同态$\varphi$是
    \begin{align*} 
    \varphi:G\to G
    \end{align*}
    则称其为自同态,记$G$所有自同态的集合为$\mathrm{End}(G)$。同理记$\mathrm{Aut}(G)$为$G$全体自同构组成的集合。
    注意到$\mathrm{Aut}(G)$组成一个群,而$\mathrm{End}(G)$是一个幺半群。这是显然的,因为同态不一定是双射,不一定可逆。
\end{definition}
\begin{definition}[同态的像与核]
    定义$\varphi(G)$为$\varphi$的像,记为$\mathrm{im} \varphi$而单位元对应的原像定义为核,也就是
    \begin{align*} 
    \ker \varphi := \{a|\varphi(a) = e\}
    \end{align*}
    不难发现,商群的单位元就是$H$,因此若要验证一个同态是单的,只需要任取$H$中的元素,验证其被映射到像空间的单位元,并且验证
    对于任意原空间中的元素,若其被映射到单位元,可以推出其属于$H$,就可以验证其单射性质;若要验证其满射,只需要取像空间中
    任意元素,说明其在同态下存在原像(被作用的元素属于原空间)即可。
\end{definition}
同态是单射并不好验证,但是我们可以通过以下的等价条件进行验证:
\begin{theorem}
$\ker \varphi = \{e\}\Leftrightarrow \varphi $单
\end{theorem}
接下来我们将要介绍同构中非常关键且基础的一个定理:
\begin{theorem}[同构基本定理]
设$\varphi$是群同态,则
\begin{align*} 
    G/\ker \varphi \cong \mathrm{im} \varphi
\end{align*}
\end{theorem}
\begin{definition}[平移]
    定义$G$上的变换
    \begin{align*} 
    L(a):G \to G
    \end{align*}
    称$L(a)$为由$a$引起的左平移,同理可以定义右平移。
\end{definition}
\begin{theorem}[凯莱定理]
任一群同构于某一集合上的变换群。
\end{theorem}
\begin{theorem}[凯莱定理(有限群)]
任一群$G$同构于对称群$S_{|G|}$的某个子群。
\end{theorem}
\begin{proof} 
    设$L(G)$是左平移的全体构成的$G$的全变换群$S(G)$的子集,定义
    \begin{align*} 
    L:G \to S(G)
    \end{align*}
    对于任意$a,b,g\in G$
    \begin{align*} 
    L(ab)g = (ab)g = a(bg) = L(a)(bg) = (L(a)L(b))g
    \end{align*}
    因此$L$是群同态。注意到$\mathrm{im}L = L(G)$,对于$L$而言,$L(a) = \mathrm{id}$ 当且仅当$a = e$,因此$\ker L = \{e\}$,因此由同态基本定理得到
    \begin{align*} 
    G\cong L(G)
    \end{align*}
    得证.
\end{proof}
\begin{remark}
    凯莱定理事实上阐述了这样一件事情:任何抽象的群与某个置换群同构。这使得我们能够具体地研究群
    的内部结构。由于变换群可以由具体的自然数的置换表示,我们的研究就会更加简单和容易。因此我们将
    $L(G)$称为$G$的左正则表示,对应地,$R(G)$被称为右正则表示。
\end{remark}
实际上,在范畴论中,我们有更广泛的结论:
\begin{theorem}[米田引理]
设$\mathcal{C}$为局部小范畴,$X\in \mathcal{O}b$。设$F$为从任意$\mathcal{C}$到$\mathbb{SET}$的函子,存在从$Nat(h_X,F)$到集合
$F(X)$的双射。且该双射为
\begin{align*} 
    \alpha \mapsto \alpha_X(\mathrm{id}_X)
\end{align*}
其逆映射把$u\in F(X)$映射到自然变换$\beta$,使得对于任意的$T\in \mathcal{O}b(\mathcal{C})$,都有
\begin{align*} 
    \beta_Y:Hom_{\mathcal{C}}(X,Y)\to F(Y)
\end{align*}
我们不会深入这一结论,因为这远远超出了本笔记的叙述“范畴”。
\end{theorem}
回到正题,我们直觉上愿意定义所谓的“典范”,以便于我们进行群结构的研究,因此我们将从原群到商群的同态记为:
\begin{definition}[典范同态]
    若$H\trianglelefteq G$,则易得
    \begin{align*} 
    \pi:G \to G/H
    \end{align*}
    是同态。这一同态被称为从$G$到$G/H$的典范同态。
\end{definition}
\begin{remark}
    容易发现,由于$H$是正规子群,因此商群存在。典范同态保证从原来的群转换为商群的过程中保持运算,使得商群
    的结构是良定义的。
\end{remark}
\begin{theorem}[第一同构定理]
$H\trianglelefteq G$,在$G$到$G/H$典范同态下
\begin{itemize}
        \item $G$的包含$H$的子群与$G/H$的子群在典范同态下一一对应;
        \item 上述一一对应下,正规子群对应正规子群;
        \item 若$K\trianglelefteq G$且$K\supseteq H$,则
              \begin{align*} 
                G/K \cong (G/H)/(K/H)
              \end{align*}
\end{itemize}
\end{theorem}
\begin{remark}
    这个定理表明,商运算保持正规性,低阶商群可以通过高阶商群构造。这一性质进一步强调了正规性的重要性。
\end{remark}

\begin{theorem}[第二同构定理]
    设$G$是群,$H\trianglelefteq G$,$K\leq G$,那么以下结论成立:
    \begin{itemize}
        \item $HK\leq G, H\cap K \trianglelefteq K$;
        \item $(HK)/H\cong K/(K\cap H)$
    \end{itemize}
\end{theorem}
\begin{remark}
    这一定理的证明是不难的。首先,只需要通过判定定理验证$HK$是子群,然后通过正规子群判定定理老老实实
    验证正规子群。证明同构还是四步法:证明Well-Defined(证明运算性质与代表元选取无关,常见手法是通过
    选取两个假设相等的元素,验证其像是否一定相等)、证明为群同态(保持运算)、证明单(核空间只有单位元)
    、证明满(每个像对应原像)。这些基本的定理基本上都按照以上步骤进行,这是基本功。
\end{remark}
\begin{remark}
    定理显然表明,积运算不保持正规性,交运算保持正规性;同时,子群积对正规子群的商同构于子群对子群交的商。
\end{remark}

\subsection{群的直积与直和}
新群的构造可以通过简单的复合进行:
\begin{definition}[直和]
    设群$G_1,G_2$,由笛卡尔积得群$G_1\times G_2$,该集合在运算
    \begin{align*} 
    (a_1,b_1)(a_2,b_2) = (a_1a_2,b_1b_2)
    \end{align*}
    下保持群结构。称为群的(外)直和,记为$G_1\oplus G_2$,$G_1$和$G_2$称为$G_1\oplus G_2$的直和因子。
\end{definition}
我们常常反过来考虑:是否可以将一个群分解为多个群的直和?我们给出如下的等价判定条件:
\begin{theorem}[群的直和分解判定]
    设$G$是群,$H,K\trianglelefteq G$,$G=HK$,那么以下结论等价:
    \begin{itemize}
        \item 映射 
        \begin{align*}
            \sigma: H\oplus K \to G 
        \end{align*}
        是同构;
        \item $G$的任意元素唯一表示为$H$和$K$元素的乘积;
        \item $G$的单位元唯一表示为$H$和$K$元素的乘积;
        \item $H\cap K$ = \{e\};
    \end{itemize}
\end{theorem}
\begin{remark}
    判定内容是直观的,这表明两个群负责不同维度的内容而不存在交集。事实上,从第四个结论就可以联想到线性代数中的
    直和分解的内容,这在群中也是相同的。事实上,这与实际我们人工智能的分类也有很大关系:若一个事物的描述是可分的,
    那么它一定是若干直和的结果。
\end{remark}
满足如上性质的$G$被称为$H$和$K$的(内)直和,记号相同。对于多个群的直和,我们同样可以证明上述四个定理,只需要使用
归纳法进行验证即可。其中注意,第四个定理为$\forall i,H_i\cap(H_1\cdots \hat{H_i}\cdots H_n)=\{e\}$,其中$\hat{H_i}$
表示去掉$H_i$。
\begin{definition}[直积]
    设$G$是群,$I$为集合的指标集(可以是无穷集合)。记
    \begin{align*}
         \prod_{i\in I}G_i
    \end{align*}
    为(外)直积。同样为集合的笛卡尔积,运算同样为按分量运算。显然,若I有限,其与直和无区别。但是在无限集情况下,二者不同。
    该集合的一个自然的子集为$\{(\cdots,a_i,\cdots)|a_i\in G_i, \forall j \ne i, a_j = e_j\}$。
\end{definition}
存在一个映射将任一群映射到直积群中,当指标集有限,直和的任意元素都可以唯一地表示为像空间中的和(线性代数中,这被称为正交基),
而无限集中则没有如上性质。

\paragraph{}终于,我们结束了群的内容,接下来我们了解环的内容。事实上,环的内容有很多与群是类似的。

\section{环}


\section{体和域}

\chapter{特殊的群}

\chapter{环与数论}

\chapter{域扩张与伽罗瓦理论}

\chapter{模与格}

\chapter{群表示论}




\end{document}