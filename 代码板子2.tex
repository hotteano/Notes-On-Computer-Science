\documentclass[12pt,a4paper,UTF16]{ctexbook}
\usepackage[top=2cm, bottom=2cm, left=2cm, right=2cm]{geometry} % 页边距
\usepackage{amsmath,amssymb}    % 数学公式
\usepackage{graphicx}           % 图片插入
\usepackage{booktabs}           % 专业表格
\usepackage{multirow}            % 表格跨行
\usepackage{listings}
\usepackage{bussproofs}
\usepackage{tikz}
\usepackage{multirow}   % 合并行
\usepackage{hhline}     % 绘制自定义水平线
\usepackage{array}      % 调整列格式
\usepackage[colorlinks=true]{hyperref} % 超链接
\usepackage{amssymb}
\usepackage{array}
\usepackage[dvipsnames]{xcolor}
\usepackage{mathpartir}
\usepackage{amsthm}
\usepackage{longtable}
\usepackage[utf8]{inputenc}
\usepackage[english]{babel}
\theoremstyle{plain}
\newtheorem{theorem}{\indent Theorem}[section]
\newtheorem{lemma}[theorem]{\indent Lemma}
\newtheorem{note}[theorem]{\indent Notation}
\newtheorem{proposition}[theorem]{\indent Proposition}
\newtheorem{corollary}[theorem]{\indent 推论}
\newtheorem{definition}{\indent Definition}[section]
\newtheorem{example}{\indent Example}[section]
\newtheorem{remark}{\indent Remark}[section]
\newenvironment{solution}{\begin{proof}[\indent\bf Solution]}{\end{proof}}
\renewcommand{\proofname}{\indent\bf Proof}
\lstset{
 language=c++, % 设置语言
 basicstyle=\ttfamily, % 设置字体族
 breaklines=true, % 自动换行
 keywordstyle=\bfseries\color{Aquamarine}, % 设置关键字为粗体,颜色为 NavyBlue
 morekeywords={}, % 设置更多的关键字,用逗号分隔
 emph={self}, % 指定强调词,如果有多个,用逗号隔开
    emphstyle=\bfseries\color{Rhodamine}, % 强调词样式设置
    commentstyle=\itshape\color{black!50!white}, % 设置注释样式,斜体,浅灰色
    stringstyle=\bfseries\color{PineGreen!90!black}, % 设置字符串样式
    columns=flexible,
    numbers=left, % 显示行号在左边
    numbersep=2em, % 设置行号的具体位置
    numberstyle=\footnotesize, % 缩小行号
    frame=single, % 边框
    framesep=1em % 设置代码与边框的距离
}

\title{\bf{代码板子:第二弹}}
\author{陈彦桥}



\begin{document}
\maketitle

\section{前言}
本册主要包含一些模板题的代码板子,主要来源是本人的做题经历。

\tableofcontents

\chapter{动态规划}
\section{如何考虑动态规划}
\subsection{考虑状态}
如何考虑题目中的状态?应该首先考虑如下内容:
\begin{itemize}
    \item 集合,也就是包含哪些元素
    \item 属性,这些元素具有的属性:最大值、最小值、数量……
\end{itemize}
其中,我们重点考虑集合的表示:首先我们需要确定集合到底是什么,原本集合的元素?选法?还是值?然后,我们
要确定我们的维度的含义。常见的$f[i][j]$可以表示:
\begin{itemize}
    \item 下标不超过i,总属性也不超过j的各种选法
    \item 左边界在i,右边界在j的元素
\end{itemize}
等等,以此类推,需要做题积累经验。
\subsection{考虑状态转移}
考虑特定的某一个状态,讨论所有转移到该状态的前置状态。此时,我们进行的是集合的划分,将当前的集合划分为若干
更小的集合,使得前面计算过的集合能够用于计算当前的状态。需要注意,我们不一定只能通过循环求出结果,也可以
使用递归的方法,也就是记忆化搜索。
\paragraph{}如何得到状态递推?或者说,如何找一些前置集合,使得能够产生递推关系?我们可以考虑比较小规模的问题,
然后进行分类讨论,以此得到一些不变的性质;或者,我们可以拿出代表对象,通过对对象位置、大小等性质的讨论和比较
进行分类,从而得到更小的子问题。总而言之,我们需要找到当前比当前维度更小的子问题,这一问题通常是通过数学计算
和分类讨论得到的。
\paragraph{}如果需要维护最大值,我们首先要将其旧值进行比较。对于背包问题,我们常常是讨论选不选当前物品,进而
将不选对应到上一个$f[i-1][j]$,将选该物品对应到$f[i-1][j-w[i]] + w[i]$。在其它问题中也可以类似地找到类似的
讨论对象。例如在数位dp中,上一个选择会影响下一个选择,但是我们只需要考虑当前选择对于选择之前状态的限制以及影响,
就可以很好的找到递推公式。
\section{背包DP}
\subsection{0-1背包问题}
\begin{lstlisting}
//二维版本
int knapsack(vector<int>& weights,vector<int>& values, int W) 
{
int n = weights.size();
//初始化略
vector<vector<int>> dp(W + 1, vector<int>(n));
for (int i = 0; i < n; ++i) {
    for (int j = weights[i]; j <= W; j++) {
        dp[i][j]=max(dp[i][j],dp[i-1][j-weights[i-1]]+values[i-1]);
    }
}
return dp[W][n - 1];
}
\end{lstlisting}
\begin{lstlisting}
//一维版本
int knapsack(vector<int>& weights,vector<int>& values, int W) 
{
int n = weights.size();
vector<int> dp(W + 1, 0);
for (int i = 0; i < n; ++i) {
    for (int w = W; w >= weights[i]; --w) {
        //逆序遍历保证每一个物品只选择一遍
        dp[w] = max(dp[w],dp[w - weights[i]] + values[i]);
    }
}
return dp[W];
}
\end{lstlisting}

\subsection{完全背包问题}
\begin{lstlisting}
//枚举版本
int knapsack(vector<int>& weights,vector<int>& values, int W) 
{
int n = weights.size();
//初始化略
vector<vector<int>> dp(W + 1, vector<int>(n + 1));
for (int i = 1; i <= n; ++i) {
    for (int j = 0; j <= W; ++j) {
        for(int k = 0; k * weights[i] <= j; ++k)
            dp[i][j]=max(dp[i][j],dp[i][j-k * weights[i]]+values[i] * k);
    }
}
return dp[W][n];
}
\end{lstlisting}
注意到,$dp[i][j] = MAX(dp[i-1][j], dp[i-1][j-w]+v, dp[i-1][j-2w]+2v, \cdots, dp[i-1][j-mw]+mv)$中,后面
的$dp[i-1][j-w]+v, dp[i-1][j-2w]+2v, \cdots, dp[i-1][j-mw]+mv$与$dp[i][j-w]$十分相似,如下:
\begin{align*} 
    dp[i][j-w] = MAX(dp[i-1][j-w],  \cdots, dp[i-1][j-(m-1)w]+(m-1)v)
\end{align*}
与上式仅差一个$v$, 而同时加一个值不影响最大值,从而注意到
\begin{align*} 
    dp[i][j] = max(dp[i-1][j], dp[i-1][j-w]+v)
\end{align*}
综上有优化代码:
\begin{lstlisting}
//优化版本
int knapsack(vector<int>& weights,vector<int>& values, int W) 
{
int n = weights.size();
//初始化略
vector<vector<int>> dp(W + 1, vector<int>(n + 1));
for (int i = 1; i <= n; ++i) {
    for (int j = 0; j <= W; ++j) {
      dp[i][j] = dp[i-1][j];
      if(j >= weights[i])dp[i][j]=max(dp[i][j],dp[i][j-weights[i]]+values[i]);
    }
}
return dp[W][n];
}
\end{lstlisting}
\subsection{多重背包问题}
\begin{lstlisting}
//枚举版本
int knapsack(vector<int>& num, vector<int>& weights,vector<int>& values, int W) 
{
int n = weights.size();
//初始化略
vector<vector<int>> dp(W + 1, vector<int>(n + 1));
for (int i = 1; i <= n; ++i) {
    for (int j = 0; j <= W; ++j) {
        for(int k = 0; k * weights[i] <= j && k <= num[i]; ++k)
            dp[i][j]=max(dp[i][j],dp[i][j-k * weights[i]]+values[i] * k);
    }
}
return dp[W][n];
}
\end{lstlisting}
我们尝试优化上面的代码,注意到$
dp[i][j] = MAX(dp[i-1][j], dp[i-1][j-w]+v, \cdots, dp[i-1][j-s[i]*w]+s[i]*v)$以及$
dp[i][j-w] = MAX(dp[i-1][j-w],  \cdots, dp[i-1][j-s[i]*w]+(s[i]-1)*v, dp[i-1][j-(s[i]+1)*w]+s[i]*v)$
这里,与完全背包不同的是,我们的第二个式子最后会多出一项$dp[i-1][j-(s[i]+1)*w]+s[i]*v$, 这是因为,此时我们有选择
的上限,使得我们无法对齐尾部这一项(第一个式子的物品数量不能再加一以对齐了)。因此,这样优化是不行的。所以我们要
考虑其它的办法。
\paragraph{}我们考虑倍增法,考虑小于等于s[i]的每个数的二进制编码,我们将“选k个物品”转化为“在数个二进制位中选任意个,且
表示的这个数的大小不超过s[i]”,每一个位的“价值”就是二的任意次幂,这样我们就可以将问题转化为“二进制位的0-1背包问题”,
而这个二进制位的长度是$log(max(s[i]))$,从而我们避免了从0到s[i]枚举每一个数字。
\begin{lstlisting}
#include<iostream>
using namespace std;
const int N=11010,M=2010;
int n,m;
int v[N],w[N];
int f[M];
int main()
{
    cin>>n>>m;
    int cnt=1;
    for(int i=1;i<=n;i++){//拆分打包
        int a,b,s;
        cin>>a>>b>>s;
        int k=1;
        while(k<=s){
            v[cnt]=a*k;
            w[cnt]=b*k;
            s-=k;
            k*=2;
            cnt++;//将s[i]个物品拆分为1个,2个,4个……分别打包为一个物品
            //最终就是在打包好的物品中进行0-1背包问题,从而保证了正确性,因为二进制
            //可以表示小于编码长度的所有整数
        }
        if(s>0){
            v[cnt]=s*a;
            w[cnt]=s*b;
            cnt++;
        }
    }
    for(int i=1;i<=cnt;i++){
        for(int j=m;j>=v[i];j--)
        {
            f[j]=max(f[j],f[j-v[i]]+w[i]);
        }
    }
    cout<<f[m]<<endl;
    return 0;
}
\end{lstlisting}
\subsection{分组背包问题}
\begin{lstlisting}
int knapsack(vector<vector<int>>& weights, vector<vector<int>>& values, vector<int>& nums,int W) 
{
int n = weights.size();
//初始化略
vector<int> dp(W + 1);
for (int i = 1; i <= n; ++i) {
    for (int j = W; j >= 0; --j) {
        for(int k = 0; k < nums[i]; ++k){
            if(weights[i][k] <= j)dp[j]=max(dp[j],dp[j-weights[i][k]]+values[i][k]);
        }//要么从该组选一个,要么不选
    }
}
return dp[W];
}
\end{lstlisting}

\section{线性dp}
\subsection{数字三角形}

数字三角形就是典型的移动类型问题,只需要计算出上一个位置的最大特征,就可以递推下一个状态。
\begin{lstlisting}
#include <iostream>
using namespace std;
const int N = 510;
int n;
int a[N][N];
int f[N][N];
int main () {
    cin >> n;
    for (int i = 1;i <= n;i++) {
        for (int j = 1;j <= i;j++) cin >> a[i][j];
    }
    for (int i = 1;i <= n;i++) f[n][i] = a[n][i];
    for (int i = n - 1;i >= 1;i--) {
        for (int j = 1;j <= i;j++) {
            f[i][j] = max (f[i + 1][j],f[i + 1][j + 1]) + a[i][j];
        }
    }
    cout << f[1][1] << endl;
    return 0;
}
\end{lstlisting}

\subsection{最长上升子序列}
\begin{lstlisting}
//朴素dp
class Solution {
public:
    int lengthOfLIS(vector<int>& nums) {
        int l=nums.size();
        vector<int>dp(l);//dp[i]表示以i结尾的递增序列长度
        dp[0]=1;
        for(int i=0;i<l;i++){
            dp[i]=1;
            for(int j=0;j<i;j++){
                if(nums[i]>nums[j]){
                    dp[i]=max(dp[i],dp[j]+1);
                }
            }
        }
        int MAX=INT_MIN;
        for(int i=0;i<l;i++){
            MAX=max(dp[i],MAX);
        }
        return MAX;
    }
};
\end{lstlisting}
\begin{lstlisting}
//线段树优化,解决间隔不超过k的递增最长子序列
class Solution {
    vector<int> max;

    void modify(int o, int l, int r, int i, int val) {
        if (l == r) {
            max[o] = val;
            return;
        }
        int m = (l + r) / 2;
        if (i <= m) modify(o * 2, l, m, i, val);
        else modify(o * 2 + 1, m + 1, r, i, val);
        max[o] = std::max(max[o * 2], max[o * 2 + 1]);
    }

    // 返回区间 [L,R] 内的最大值
    int query(int o, int l, int r, int L, int R) { // L 和 R 在整个递归过程中均不变,将其大写,视作常量
        if (L <= l && r <= R) return max[o];
        int res = 0;
        int m = (l + r) / 2;
        if (L <= m) res = query(o * 2, l, m, L, R);
        if (R > m) res = std::max(res, query(o * 2 + 1, m + 1, r, L, R));
        return res;
    }

public:
    int lengthOfLIS(vector<int> &nums, int k) {
        int u = *max_element(nums.begin(), nums.end());
        max.resize(u * 4);
        for (int x: nums) {
            if (x == 1) modify(1, 1, u, 1, 1);
            else {
                int res = 1 + query(1, 1, u, std::max(x - k, 1), x - 1);
                modify(1, 1, u, x, res);
            }
        }
        return max[1];
    }
};
\end{lstlisting}
\newpage
\subsection{最长公共子序列}
\begin{lstlisting}
class Solution {
public:
    int longestCommonSubsequence(string text1, string text2) {
        int l=text1.length();
        int r=text2.length();
        int dp[l+1][r+1];//dp[i][j]表示,第一个字符串前i个和第二个字符串前j个字母的最长公共序列
        for(int i=0;i<=l;i++)
        {
            dp[i][0]=0;
        }
        for(int i=0;i<=r;i++)
        {
            dp[0][i]=0;
        }
        for(int i=1;i<=l;i++)
        {
            for(int j=1;j<=r;j++)
            {
                if(text1[i-1]==text2[j-1]){dp[i][j]=dp[i-1][j-1]+1;}
                else{dp[i][j]=max(dp[i-1][j],dp[i][j-1]);}
            }
        }
        return dp[l][r];
    }
};
\end{lstlisting}
\subsection{字符串编辑距离}
\begin{lstlisting}
class Solution {
public:
    int minDistance(string word1, string word2) {
        int n = word1.length();
        int m = word2.length();

        // 有一个字符串为空串
        if (n * m == 0) return n + m;

        // DP 数组
        vector<vector<int>> D(n + 1, vector<int>(m + 1));

        // 边界状态初始化
        for (int i = 0; i < n + 1; i++) {
            D[i][0] = i;
        }
        for (int j = 0; j < m + 1; j++) {
            D[0][j] = j;
        }

        // 计算所有 DP 值
        // 对于A的前i个字符和B的前j个字符,若最后一个字符相同,那么直接用前i-1和前j-1的编辑距离
        // 如果最后一个字符不同,那么只需要修改一个字符即可
        // 对于其它情况,至少需要增加一次编辑距离(也就是插入一个字符)
        for (int i = 1; i < n + 1; i++) {
            for (int j = 1; j < m + 1; j++) {
                int left = D[i - 1][j] + 1;
                int down = D[i][j - 1] + 1;
                int left_down = D[i - 1][j - 1]; //暂存防止意外改变原始值
                if (word1[i - 1] != word2[j - 1]) left_down += 1;
                D[i][j] = min({left, down, left_down});

            }
        }
        return D[n][m];
    }
};
\end{lstlisting}
\begin{remark}
    可以发现,两个字符串类型的dp问题通常可以使用“A的前i个XXX和B的前j个XXX”作为状态进行递推。
\end{remark}


\section{区间dp}
\subsection{合并石子}
\begin{lstlisting}
#include <iostream>
#include <cstring>
using namespace std;
const int N = 307;

int a[N], s[N];
int f[N][N];

int main() {
    int n;
    cin >> n;

    for (int i = 1; i <= n; i ++) {
        cin >> a[i];
        s[i] += s[i - 1] + a[i];
    }

    memset(f, 0x3f, sizeof f);
    // 区间 DP 枚举套路:长度+左端点 
    for (int len = 1; len <= n; len ++) { // len表示[i, j]的元素个数
        for (int i = 1; i + len - 1 <= n; i ++) {
            int j = i + len - 1; // 自动得到右端点
            if (len == 1) {
                f[i][j] = 0;  // 边界初始化
                continue;
            }

            for (int k = i; k <= j - 1; k ++) { // 必须满足k + 1 <= j
                f[i][j] = min(f[i][j], f[i][k] + f[k + 1][j] + s[j] - s[i - 1]);
            }
        }
    }
    cout << f[1][n] << endl;
    return 0;
}
\end{lstlisting}

\section{计数dp}
\subsection{整数划分}
\begin{lstlisting} 
// f[i][j] = f[i - 1][j] + f[i][j - i]
#include <iostream>

using namespace std;

const int N = 1e3 + 7, mod = 1e9 + 7;

int f[N][N];

int main() {
    int n;
    cin >> n;

    for (int i = 0; i <= n; i ++) {
        f[i][0] = 1; // 容量为0时,前 i 个物品全不选也是一种方案
    }

    for (int i = 1; i <= n; i ++) {
        for (int j = 0; j <= n; j ++) {
            f[i][j] = f[i - 1][j] % mod; // 特殊 f[0][0] = 1
            if (j >= i) f[i][j] = (f[i - 1][j] + f[i][j - i]) % mod;
        }
    }

    cout << f[n][n] << endl;
}
\end{lstlisting}

\begin{lstlisting}
#include <iostream>

using namespace std;

const int N = 1e3 + 7, mod = 1e9 + 7;

int f[N];

int main() {
    int n;
    cin >> n;


    f[0] = 1; // 容量为0时,前 i 个物品全不选也是一种方案

    for (int i = 1; i <= n; i ++) {
        for (int j = i; j <= n; j ++) {
            f[j] = (f[j] + f[j - i]) % mod;
        }
    }

    cout << f[n] << endl;
}
\end{lstlisting}
实际上,这就是著名的分拆数的计算方法。目前没有一个闭式公式用于计算这个数,但是可以通过渐进的方式估计
该数的大小。

\section{数位统计dp}
\subsection{数位统计}
关键在于分情况讨论,根据题目中的限制条件列出可能的情况进行递归。不妨定义以下两种情况
\begin{itemize}
    \item Dominate(压制):高位恒小于上限,低位任取;
    \item Equal(等于):高位等于上限,低位讨论;
\end{itemize}
\begin{lstlisting}
#include <iostream>
#include <algorithm>
#include <cmath>

using namespace std;

int get(int n) { // 求数n的位数
    int res = 0;
    while (n) res ++, n /= 10;
    return res;
}

int count(int n, int i) { // 求从1到数n中数i出现的次数
    int res = 0, dgt = get(n);

    for (int j = 1; j <= dgt; ++ j) {
    /* p为当前遍历位次(第j位)的数大小 <10^(右边的数的位数)>, Ps:从左往右(从高位到低位)
        l为第j位的左边的数,r为右边的数,dj为第j位上的数 */
        int p = pow(10, dgt - j), l = n / p / 10, r = n % p, dj = n / p % 10;

   // ps:下文的xxx、yyy均只为使读者眼熟,并不严格只是三位数啊~ 然后后续的...就代表省略的位数啦~
        /* 求要选的数在i的左边的数小于l的情况:
(即视频中的xxx1yyy中的xxx的选法) --->
    1)、当i不为0时 xxx : 0...0 ~ l - 1, 即 l * (右边的数的位数) == l * p 种选法
    2)、当i为0时 由于不能有前导零 故xxx: 0....1 ~ l - 1, 
                    即 (l-1) * (右边的数的位数) == (l-1) * p 种选法 */
        if (i) res += l * p;
        else res += (l - 1) * p;

        /* 求要选的数在i的左边的数等于l的情况:(即视频中的xxx == l 时)
    (即视频中的xxx1yyy中的yyy的选法)--->
                        1)、i > dj时 0种选法
                        2)、i == dj时 yyy : 0...0 ~ r 即 r + 1 种选法
                        3)、i < dj时 yyy : 0...0 ~ 9...9 即 10^(右边的数的位数) == p 种选法 */
        if (i == dj) res += r + 1;
        if (i < dj) res += p;
    }

    return res; // 返回结果
}

int main() {
    int a, b;
    while (cin >> a >> b, a) { // 输入处理,直到输入为0停止
        if (a > b) swap(a, b); // 预处理-->让a为较小值,b为较大值
        for (int i = 0; i <= 9; ++ i) cout << count(b, i) - count(a - 1, i) << ' '; 
        // 输出每一位数字(0 ~ 9)分别在[a,b]中出现的次数<利用前缀和思想:[l,r]的和=s[r] - s[l - 1]>
        cout << endl; //换行
    }

    return 0; 
}
\end{lstlisting}

\chapter{字符串专题}

\subsection{KMP}
\begin{lstlisting}
// 在文本串 text 中查找模式串 pattern,返回所有成功匹配的位置(pattern[0] 在 text 中的下标)
vector<int> kmp(const string& text, const string& pattern) {
    int m = pattern.size();
    vector<int> pi(m);
    int cnt = 0;
    for (int i = 1; i < m; i++) {
        char b = pattern[i];
        while (cnt && pattern[cnt] != b) {
            cnt = pi[cnt - 1];
        }
        if (pattern[cnt] == b) {
            cnt++;
        }
        pi[i] = cnt;
    }

    vector<int> pos;
    cnt = 0;
    for (int i = 0; i < text.size(); i++) {
        char b = text[i];
        while (cnt && pattern[cnt] != b) {
            cnt = pi[cnt - 1];
        }
        if (pattern[cnt] == b) {
            cnt++;
        }
        if (cnt == m) {
            pos.push_back(i - m + 1);
            cnt = pi[cnt - 1];
        }
    }
    return pos;
}
\end{lstlisting}

\subsection{Manacher算法}
本算法用于高效解决回文串判定。
\begin{lstlisting}
class Solution {
public:
    int expand(const string& s, int left, int right) {
        while (left >= 0 && right < s.size() && s[left] == s[right]) {
            --left;
            ++right;
        }
        return (right - left - 2) / 2;
    }

    string longestPalindrome(string s) {
        int start = 0, end = -1;
        string t = "#";
        for (char c: s) {
            t += c;
            t += '#';
        }
        t += '#';
        s = t;

        vector<int> arm_len;
        int right = -1, j = -1;
        for (int i = 0; i < s.size(); ++i) {
            int cur_arm_len;
            if (right >= i) {
                int i_sym = j * 2 - i;
                int min_arm_len = min(arm_len[i_sym], right - i);
                cur_arm_len = expand(s, i - min_arm_len, i + min_arm_len);
            } else {
                cur_arm_len = expand(s, i, i);
            }
            arm_len.push_back(cur_arm_len);
            if (i + cur_arm_len > right) {
                j = i;
                right = i + cur_arm_len;
            }
            if (cur_arm_len * 2 + 1 > end - start) {
                start = i - cur_arm_len;
                end = i + cur_arm_len;
            }
        }

        string ans;
        for (int i = start; i <= end; ++i) {
            if (s[i] != '#') {
                ans += s[i];
            }
        }
        return ans;
    }
};
\end{lstlisting}

\chapter{进阶数据结构}

\subsection{ST表}
\begin{lstlisting}
#include<cstdio>
#include<cmath>
using namespace std;
const int N=1e5+5;
int n,m;
int a[N];
int st[N][25];
int main()
{
	scanf("%d%d",&n,&m);
	for(int i=1;i<=n;i++)scanf("%d",&a[i]);//读入数据
	for(int i=1;i<=n;i++)st[i][0]=a[i];//初始化,区间长度为0
	for(int j=1;1<<j<=n;j++){ //j的边界就是右边
		for(int i=1;i+(1<<j)<=n+1;i++){ //i+j的边界是i+2^{j-1}+1
			st[i][j]=max(st[i][j-1],st[i+(1<<(j-1))][j-1]);//转移处理ST表
        }
    }//读入预处理阶段,构建ST表
	while(m--)
	{
		int l,r;
		scanf("%d%d",&l,&r);
		int k=__lg(r-l+1);//将区间长度分拆为两个2^k
		printf("%d\n",max(st[l][k],st[r-(1<<k)+1][k])); //利用公式计算出左端点
	} 
	return 0;
} 
\end{lstlisting}

\subsection{字典树}
\begin{lstlisting}
struct Node {
    Node* son[26]{};
    bool end = false;
};

class Trie {
    Node* root = new Node();

    int find(string word) {
        Node* cur = root;
        for (char c : word) {
            c -= 'a';
            if (cur->son[c] == nullptr) { // 道不同,不相为谋
                return 0;
            }
            cur = cur->son[c];
        }
        // 走过同样的路(2=完全匹配,1=前缀匹配)
        return cur->end ? 2 : 1;
    }

    void destroy(Node* node) {
        if (node == nullptr) {
            return;
        }
        for (Node* son : node->son) {
            destroy(son);
        }
        delete node;
    }

public:
    ~Trie() {
        destroy(root);
    }

    void insert(string word) {
        Node* cur = root;
        for (char c : word) {
            c -= 'a';
            if (cur->son[c] == nullptr) { // 无路可走?
                cur->son[c] = new Node(); // new 出来!
            }
            cur = cur->son[c];
        }
        cur->end = true;
    }

    bool search(string word) {
        return find(word) == 2;
    }

    bool startsWith(string prefix) {
        return find(prefix) != 0;
    }
};
\end{lstlisting}

\subsection{AC自动机}
AC自动机是多字符串匹配的经典算法,是KMP思想在多字符上的运用。实际上,KMP就是链表形式的
AC自动机。事实上,我们在这个算法中的核心目的,就是找到当前失配字符串后缀对应的最长前缀。
该算法的关键在于建立fail数组,其中,某一个结点x的fail数组为:x->fa->fail->(若存在fa到x的字母,x的fail指向它)->否则继续查找fail。
\paragraph{}为了压缩路径,我们可以直接设置转移表,每一个结点对应读到字符的转移路径。如何设置呢?我们
只需要将当前节点的fail结点读到字符的fail转移结点一致即可。
\begin{lstlisting}
typedef struct TrieNode {
    vector<TrieNode *> children;
    bool isEnd;
    TrieNode *fail;
    TrieNode() {
        this->children = vector<TrieNode *>(26, nullptr);
        this->isEnd = false;
        this->fail = nullptr;
    }
};

class StreamChecker {
public:
    TrieNode *root;
    TrieNode *temp;
    StreamChecker(vector<string>& words) {
        root = new TrieNode();
        for (string &word : words) {
            TrieNode *cur = root;
            for (int i = 0; i < word.size(); i++) {
                int index = word[i] - 'a';
                if (cur->children[index] == nullptr) {
                    cur->children[index] = new TrieNode();
                }
                cur = cur->children[index];
            }
            cur->isEnd = true;
        }
        root->fail = root;
        queue<TrieNode *> q;
        for (int i = 0; i < 26; i++) {
            if(root->children[i] != nullptr) {
                root->children[i]->fail = root;
                q.emplace(root->children[i]);
            } else {
                root->children[i] = root;
            }
        }
        while (!q.empty()) {
            TrieNode *node = q.front();
            q.pop();
            node->isEnd = node->isEnd || node->fail->isEnd;
            for (int i = 0; i < 26; i++) {
                if(node->children[i] != nullptr) {
                    node->children[i]->fail = node->fail->children[i];
                    q.emplace(node->children[i]);
                } else {
                    node->children[i] = node->fail->children[i];
                }
            }
        }

        temp = root;
    }
    
    bool query(char letter) {
        temp = temp->children[letter - 'a'];
        return temp->isEnd;
    }
};
\end{lstlisting}

\subsection{树状数组}
\begin{lstlisting}
class NumArray {
private:
    vector<int> nums;
    vector<int> tree;

    int prefixSum(int i) {
        int s = 0;
        for (; i > 0; i &= i - 1) { // i -= i & -i 的另一种写法,等价于lowbit;访问上一节点,如i=13,则迭代i=13-1 = 12,恰好是9-12这一段;
                                    //继续访问,i=12-4=8,恰好是最左边的一段
            s += tree[i];
        }
        return s;
    }

public:
    NumArray(vector<int> &nums) : nums(nums), tree(nums.size() + 1) {
        for (int i = 1; i <= nums.size(); i++) {
            tree[i] += nums[i - 1];
            int nxt = i + (i & -i); // 下一个关键区间的右端点,例如3,对应地就是3+(3&-3)=3+1=4
            if (nxt <= nums.size()) {
                tree[nxt] += tree[i];//关键区间加上当前位置的数
            }
        }
    }//倍增法,用等长数组维护区间信息,访问下一关键结点,迭代i=i+lowbit(i)即可

    void update(int index, int val) {
        int delta = val - nums[index];//储存更新差值,从下到上传递
        nums[index] = val;//先更新值
        for (int i = index + 1; i < tree.size(); i += i & -i) {
            tree[i] += delta;//注意,迭代从index + 1开始,因为原数组从零开始
        }
    }

    int sumRange(int left, int right) {
        return prefixSum(right + 1) - prefixSum(left);
    }
};
\end{lstlisting}

\subsection{线段树}
\begin{lstlisting}
template<typename T>
class SegmentTree {
    // 注:也可以去掉 template<typename T>,改在这里定义 T
    // using T = pair<int, int>;

    int n;
    vector<T> tree;

    // 合并两个 val
    T merge_val(T a, T b) const {
        return max(a, b); // **根据题目修改**
    }

    // 合并左右儿子的 val 到当前节点的 val
    void maintain(int node) {
        tree[node] = merge_val(tree[node * 2], tree[node * 2 + 1]);
    }

    // 用 a 初始化线段树
    // 时间复杂度 O(n)
    void build(const vector<T>& a, int node, int l, int r) {
        if (l == r) { // 叶子
            tree[node] = a[l]; // 初始化叶节点的值
            return;
        }
        int m = (l + r) / 2;
        build(a, node * 2, l, m); // 初始化左子树
        build(a, node * 2 + 1, m + 1, r); // 初始化右子树
        maintain(node);
    }

    void update(int node, int l, int r, int i, T val) {
        if (l == r) { // 叶子(到达目标)
            // 如果想直接替换的话,可以写 tree[node] = val
            tree[node] = merge_val(tree[node], val);
            return;
        }
        int m = (l + r) / 2;
        if (i <= m) { // i 在左子树
            update(node * 2, l, m, i, val);
        } else { // i 在右子树
            update(node * 2 + 1, m + 1, r, i, val);
        }
        maintain(node);
    }

    T query(int node, int l, int r, int ql, int qr) const {
        if (ql <= l && r <= qr) { // 当前子树完全在 [ql, qr] 内
            return tree[node];
        }
        int m = (l + r) / 2;
        if (qr <= m) { // [ql, qr] 在左子树
            return query(node * 2, l, m, ql, qr);
        }
        if (ql > m) { // [ql, qr] 在右子树
            return query(node * 2 + 1, m + 1, r, ql, qr);
        }
        T l_res = query(node * 2, l, m, ql, qr);
        T r_res = query(node * 2 + 1, m + 1, r, ql, qr);
        return merge_val(l_res, r_res);
    }

public:
    // 线段树维护一个长为 n 的数组(下标从 0 到 n-1),元素初始值为 init_val
    SegmentTree(int n, T init_val) : SegmentTree(vector<T>(n, init_val)) {}

    // 线段树维护数组 a
    SegmentTree(const vector<T>& a) : n(a.size()), tree(2 << bit_width(a.size() - 1)) {
        build(a, 1, 0, n - 1);
    }

    // 更新 a[i] 为 merge_val(a[i], val)
    // 时间复杂度 O(log n)
    void update(int i, T val) {
        update(1, 0, n - 1, i, val);
    }

    // 返回用 merge_val 合并所有 a[i] 的计算结果,其中 i 在闭区间 [ql, qr] 中
    // 时间复杂度 O(log n)
    T query(int ql, int qr) const {
        return query(1, 0, n - 1, ql, qr);
    }

    // 获取 a[i] 的值
    // 时间复杂度 O(log n)
    T get(int i) const {
        return query(1, 0, n - 1, i, i);
    }
};

int main() {
    SegmentTree t(8, 0LL); // 如果这里写 0LL,那么 SegmentTree 存储的就是 long long 数据
    t.update(0, 1LL << 60);
    cout << t.query(0, 7) << endl;

    vector<int> nums = {3, 1, 4, 1, 5, 9, 2, 6};
    // 注意:如果要让 SegmentTree 存储 long long 数据,需要传入 vector<long long>
    SegmentTree t2(nums); // 这里 SegmentTree 存储的是 int 数据
    cout << t2.query(0, 7) << endl;
    return 0;
}
\end{lstlisting}

\subsection{Splay树}


\end{document}

