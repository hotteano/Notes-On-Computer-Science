\documentclass[12pt,a4paper,UTF16]{ctexbook}
\usepackage[top=2cm, bottom=3cm, left=2.5cm, right=2.5cm]{geometry} % 页边距
\usepackage{amsmath,amssymb}    % 数学公式
\usepackage{graphicx}           % 图片插入
\usepackage{booktabs}           % 专业表格
\usepackage{multirow}            % 表格跨行
\usepackage{listings}
\usepackage{bussproofs}
\usepackage{tikz}
\usepackage{multirow}   % 合并行
\usepackage{hhline}     % 绘制自定义水平线
\usepackage{array}      % 调整列格式
\usepackage[colorlinks=true]{hyperref} % 超链接
\usepackage{amssymb}
\usepackage{array}
\usepackage[dvipsnames]{xcolor}
\usepackage{mathpartir}
\usepackage{amsthm}
\usepackage{longtable}
\usepackage[utf8]{inputenc}
\usepackage[english]{babel}
\theoremstyle{plain}
\newtheorem{theorem}{\indent Theorem}[section]
\newtheorem{lemma}[theorem]{\indent Lemma}
\newtheorem{note}[theorem]{\indent Notation}
\newtheorem{proposition}[theorem]{\indent Proposition}
\newtheorem{corollary}[theorem]{\indent 推论}
\newtheorem{definition}{\indent Definition}[section]
\newtheorem{example}{\indent Example}[section]
\newtheorem{remark}{\indent Remark}[section]
\newenvironment{solution}{\begin{proof}[\indent\bf Solution]}{\end{proof}}
\renewcommand{\proofname}{\indent\bf Proof}
\lstset{
 language=c++, % 设置语言
 basicstyle=\ttfamily, % 设置字体族
 breaklines=true, % 自动换行
 keywordstyle=\bfseries\color{Aquamarine}, % 设置关键字为粗体,颜色为 NavyBlue
 morekeywords={}, % 设置更多的关键字,用逗号分隔
 emph={self}, % 指定强调词,如果有多个,用逗号隔开
    emphstyle=\bfseries\color{Rhodamine}, % 强调词样式设置
    commentstyle=\itshape\color{black!50!white}, % 设置注释样式,斜体,浅灰色
    stringstyle=\bfseries\color{PineGreen!90!black}, % 设置字符串样式
    columns=flexible,
    numbers=left, % 显示行号在左边
    numbersep=2em, % 设置行号的具体位置
    numberstyle=\footnotesize, % 缩小行号
    frame=single, % 边框
    framesep=1em % 设置代码与边框的距离
}

\title{\bf{代码板子:第一弹}}
\author{陈彦桥}



\begin{document}
\maketitle

\section{前言}
本册主要包含一些比较基础的代码板子,例如基础数据结构、基础优化操作、基础图论算法和搜索、基础数学算法等。
一些进阶算法的内容请见后续。

\tableofcontents


\chapter{基础优化板子}
\section{排序}
\subsection{快排}
\begin{lstlisting}
void quick_sort(int q[], int l, int r)
{
    if (l >= r) return;

    int i = l - 1, j = r + 1, x = q[l + r >> 1];
    while (i < j)
    {
        do i ++ ; while (q[i] < x);
        do j -- ; while (q[j] > x);
        if (i < j) swap(q[i], q[j]);
    }
    quick_sort(q, l, j), quick_sort(q, j + 1, r);
}
\end{lstlisting}
\subsection{归并}
\begin{lstlisting}
void merge_sort(int q[], int l, int r)
{
    if (l >= r) return;

    int mid = l + r >> 1;
    merge_sort(q, l, mid);
    merge_sort(q, mid + 1, r);

    int k = 0, i = l, j = mid + 1;
    while (i <= mid && j <= r)
        if (q[i] <= q[j]) tmp[k ++ ] = q[i ++ ];
        else tmp[k ++ ] = q[j ++ ];

    while (i <= mid) tmp[k ++ ] = q[i ++ ];
    while (j <= r) tmp[k ++ ] = q[j ++ ];

    for (i = l, j = 0; i <= r; i ++, j ++ ) q[i] = tmp[j];
}
\end{lstlisting}

\section{二分}
\subsection{整数二分}
\begin{lstlisting}
bool check(int x) {/* ... */} // 检查x是否满足某种性质

// 区间[l, r]被划分成[l, mid]和[mid + 1, r]时使用:
int bsearch_1(int l, int r)
{
    while (l < r)
    {
        int mid = l + r >> 1;
        if (check(mid)) r = mid;    // check()判断mid是否满足性质
        else l = mid + 1;
    }
    return l;
}
// 区间[l, r]被划分成[l, mid - 1]和[mid, r]时使用:
int bsearch_2(int l, int r)
{
    while (l < r)
    {
        int mid = l + r + 1 >> 1;
        if (check(mid)) l = mid;
        else r = mid - 1;
    }
    return l;
}
\end{lstlisting}
\subsection{浮点数二分}
\begin{lstlisting}
bool check(double x) {/* ... */} // 检查x是否满足某种性质

double bsearch_3(double l, double r)
{
    const double eps = 1e-6;   // eps 表示精度,取决于题目对精度的要求
    while (r - l > eps)
    {
        double mid = (l + r) / 2;
        if (check(mid)) r = mid;
        else l = mid;
    }
    return l;
}
\end{lstlisting}
\section{高精度}
\subsection{加}
\begin{lstlisting}
// C = A + B, A >= 0, B >= 0
vector<int> add(vector<int> &A, vector<int> &B)
{
    if (A.size() < B.size()) return add(B, A);

    vector<int> C;
    int t = 0;
    for (int i = 0; i < A.size(); i ++ )
    {
        t += A[i];
        if (i < B.size()) t += B[i];
        C.push_back(t % 10);
        t /= 10;
    }

    if (t) C.push_back(t);
    return C;
}
\end{lstlisting}
\subsection{减}
\begin{lstlisting}
// C = A - B, 满足A >= B, A >= 0, B >= 0
vector<int> sub(vector<int> &A, vector<int> &B)
{
    vector<int> C;
    for (int i = 0, t = 0; i < A.size(); i ++ )
    {
        t = A[i] - t;
        if (i < B.size()) t -= B[i];
        C.push_back((t + 10) % 10);
        if (t < 0) t = 1;
        else t = 0;
    }

    while (C.size() > 1 && C.back() == 0) C.pop_back();
    return C;
}
\end{lstlisting}
\subsection{高精度乘低精度}
\begin{lstlisting}
// C = A * b, A >= 0, b >= 0
vector<int> mul(vector<int> &A, int b)
{
    vector<int> C;

    int t = 0;
    for (int i = 0; i < A.size() || t; i ++ )
    {
        if (i < A.size()) t += A[i] * b;
        C.push_back(t % 10);
        t /= 10;
    }

    while (C.size() > 1 && C.back() == 0) C.pop_back();

    return C;
}
\end{lstlisting}
\subsection{高精度除低精度}
\begin{lstlisting}
// A / b = C ... r, A >= 0, b > 0
vector<int> div(vector<int> &A, int b, int &r)
{
    vector<int> C;
    r = 0;
    for (int i = A.size() - 1; i >= 0; i -- )
    {
        r = r * 10 + A[i];
        C.push_back(r / b);
        r %= b;
    }
    reverse(C.begin(), C.end());
    while (C.size() > 1 && C.back() == 0) C.pop_back();
    return C;
}
\end{lstlisting}
\newpage
\section{前缀和与差分}
\subsection{前缀和}
\begin{lstlisting}
\\一维前缀和
S[i] = a[1] + a[2] + ... a[i]
a[l] + ... + a[r] = S[r] - S[l - 1]
\\二维前缀和
S[i][j] = 第i行j列格子左上部分所有元素的和
以(x1, y1)为左上角,(x2, y2)为右下角的子矩阵的和为:
S[x2][y2] - S[x1 - 1][y2] - S[x2][y1 - 1] + S[x1 - 1][y1 - 1]
\end{lstlisting}
\subsection{差分}
\begin{lstlisting}
\\给区间内每个数加上c
B[l] += c, B[r + 1] -= c
\\给以(x1, y1)为左上角,(x2, y2)为右下角的子矩阵中的所有元素加上c
S[x1][y1] += c, S[x2 + 1][y1] -= c, S[x1][y2 + 1] -= c, S[x2 + 1][y2 + 1] += c
\end{lstlisting}

\section{位运算}
\subsection{求n的第k位}
\begin{lstlisting}
n >> k & 1
\end{lstlisting}
\subsection{返回n的最后一位1}
\begin{lstlisting}
lowbit(n) = n & -n
\end{lstlisting}
\newpage
\subsection{n位格雷码}
\begin{lstlisting}
//第i个n位格雷码为
i ^ (i / 2)
\end{lstlisting}

\section{双指针}
\begin{lstlisting}
for (int i = 0, j = 0; i < n; i ++ )
{
    while (j < i && check(i, j)) j ++ ;

    // 具体问题的逻辑
}
\end{lstlisting}

\section{离散化}
\begin{lstlisting}
vector<int> alls; // 存储所有待离散化的值
sort(alls.begin(), alls.end()); // 将所有值排序
alls.erase(unique(alls.begin(), alls.end()), alls.end());   // 去掉重复元素

// 二分求出x对应的离散化的值
int find(int x) // 找到第一个大于等于x的位置
{
    int l = 0, r = alls.size() - 1;
    while (l < r)
    {
        int mid = l + r >> 1;
        if (alls[mid] >= x) r = mid;
        else l = mid + 1;
    }
    return r + 1; // 映射到1, 2, ...n
}
\end{lstlisting}

\section{区间合并}
\begin{lstlisting}
// 将所有存在交集的区间合并
void merge(vector<PII> &segs)
{
    vector<PII> res;

    sort(segs.begin(), segs.end());

    int st = -2e9, ed = -2e9;
    for (auto seg : segs)
        if (ed < seg.first)
        {
            if (st != -2e9) res.push_back({st, ed});
            st = seg.first, ed = seg.second;
        }
        else ed = max(ed, seg.second);

    if (st != -2e9) res.push_back({st, ed});

    segs = res;
}
\end{lstlisting}

\chapter{数据结构}
\section{链表}
\subsection{单链表}
\begin{lstlisting}
// head存储链表头,e[]存储节点的值,ne[]存储节点的next指针,idx表示当前用到了哪个节点
int head, e[N], ne[N], idx;

// 初始化
void init()
{
    head = -1;
    idx = 0;
}

// 在链表头插入一个数a
void insert(int a)
{
    e[idx] = a, ne[idx] = head, head = idx ++ ;
}

// 将头结点删除,需要保证头结点存在
void remove()
{
    head = ne[head];
}
\end{lstlisting}
\subsection{双链表}
\begin{lstlisting}
// e[]表示节点的值,l[]表示节点的左指针,r[]表示节点的右指针,idx表示当前用到了哪个节点
int e[N], l[N], r[N], idx;

// 初始化
void init()
{
    //0是左端点,1是右端点
    r[0] = 1, l[1] = 0;
    idx = 2;
}

// 在节点a的右边插入一个数x
void insert(int a, int x)
{
    e[idx] = x;
    l[idx] = a, r[idx] = r[a];
    l[r[a]] = idx, r[a] = idx ++ ;
}

// 删除节点a
void remove(int a)
{
    l[r[a]] = l[a];
    r[l[a]] = r[a];
}
\end{lstlisting}

\section{栈}
\subsection{一般栈}
\begin{lstlisting}
// tt表示栈顶
int stk[N], tt = 0;

// 向栈顶插入一个数
stk[ ++ tt] = x;

// 从栈顶弹出一个数
tt -- ;

// 栈顶的值
return stk[tt];

// 判断栈是否为空,如果 tt > 0,则表示不为空
if (tt > 0){
    return true;
}
else{
    return false;
}
\end{lstlisting}
\subsection{单调栈}
\begin{lstlisting}
//实现代码与上述代码类似,但是维护其单调性质;用于求最近比当前数大/小的数
int tt = 0;
for (int i = 1; i <= n; i ++ )
{
    while (tt && check(stk[tt], i)) tt -- ;
    stk[ ++ tt] = i;
}
\end{lstlisting}
\section{队列}
\subsection{一般队列}
\begin{lstlisting}
// hh 表示队头,tt表示队尾
int q[N], hh = 0, tt = -1;

// 向队尾插入一个数
q[ ++ tt] = x;

// 从队头弹出一个数
hh ++ ;

// 队头的值
q[hh];

// 判断队列是否为空,如果 hh <= tt,则表示不为空
if (hh <= tt)
{

}
\end{lstlisting}
\subsection{循环队列}
\begin{lstlisting}
// hh 表示队头,tt表示队尾的后一个位置
int q[N], hh = 0, tt = 0;

// 向队尾插入一个数
q[tt ++ ] = x;
if (tt == N) tt = 0;

// 从队头弹出一个数
hh ++ ;
if (hh == N) hh = 0;

// 队头的值
q[hh];

// 判断队列是否为空,如果hh != tt,则表示不为空
if (hh != tt)
{

}
\end{lstlisting}
\subsection{单调队列}
\begin{lstlisting}
//维护方法;滑动窗口最小值
int hh = 0, tt = -1;
for (int i = 0; i < n; i ++ )
{
    while (hh <= tt && check_out(q[hh])) hh ++ ;  // 判断队头是否滑出窗口
    while (hh <= tt && check(q[tt], i)) tt -- ;
    q[ ++ tt] = i;
}
\end{lstlisting}

\section{并查集}
\subsection{朴素并查集}
\begin{lstlisting}
int p[N]; //存储每个点的祖宗节点

// 返回x的祖宗节点
int find(int x)
{
    if (p[x] != x) p[x] = find(p[x]);
    return p[x];
}

// 初始化,假定节点编号是1~n
for (int i = 1; i <= n; i ++ ) p[i] = i;

// 合并a和b所在的两个集合:
p[find(a)] = find(b);

\end{lstlisting}
\subsection{维护size的并查集(按秩合并)}
\begin{lstlisting}
int p[N], size[N];
//p[]存储每个点的祖宗节点, size[]只有祖宗节点的有意义,表示祖宗节点所在集合中的点的数量

// 返回x的祖宗节点
int find(int x)
{
    if (p[x] != x) p[x] = find(p[x]);
    return p[x];
}

// 初始化,假定节点编号是1~n
for (int i = 1; i <= n; i ++ )
{
    p[i] = i;
    size[i] = 1;
}

// 合并a和b所在的两个集合:
size[find(b)] += size[find(a)];
p[find(a)] = find(b);

\end{lstlisting}
\subsection{维护到祖宗节点距离的并查集}
\begin{lstlisting}
int p[N], d[N];
//p[]存储每个点的祖宗节点, d[x]存储x到p[x]的距离

// 返回x的祖宗节点
int find(int x)
{
    if (p[x] != x)
    {
        int u = find(p[x]);
        d[x] += d[p[x]];
        p[x] = u;
    }
    return p[x];
}

// 初始化,假定节点编号是1~n
for (int i = 1; i <= n; i ++ )
{
    p[i] = i;
    d[i] = 0;
}

// 合并a和b所在的两个集合:
p[find(a)] = find(b);
d[find(a)] = distance; // 根据具体问题,初始化find(a)的偏移量
\end{lstlisting}

\section{堆}
\begin{lstlisting}
// h[N]存储堆中的值, h[1]是堆顶,x的左儿子是2x, 右儿子是2x + 1
// ph[k]存储第k个插入的点在堆中的位置
// hp[k]存储堆中下标是k的点是第几个插入的
int h[N], ph[N], hp[N], size;

// 交换两个点,及其映射关系
void heap_swap(int a, int b)
{
    swap(ph[hp[a]],ph[hp[b]]);
    swap(hp[a], hp[b]);
    swap(h[a], h[b]);
}

void down(int u)
{
    int t = u;
    if (u * 2 <= size && h[u * 2] < h[t]) t = u * 2;
    if (u * 2 + 1 <= size && h[u * 2 + 1] < h[t]) t = u * 2 + 1;
    if (u != t)
    {
        heap_swap(u, t);
        down(t);
    }
}

void up(int u)
{
    while (u / 2 && h[u] < h[u / 2])
    {
        heap_swap(u, u / 2);
        u >>= 1;
    }
}

// O(n)建堆
for (int i = n / 2; i; i -- ) down(i);
\end{lstlisting}

\section{哈希}
\subsection{一般哈希}
\begin{lstlisting}
//拉链法
int h[N], e[N], ne[N], idx;

// 向哈希表中插入一个数
void insert(int x)
{
    int k = (x % N + N) % N;
    e[idx] = x;
    ne[idx] = h[k];
    h[k] = idx ++ ;
}

// 在哈希表中查询某个数是否存在
bool find(int x)
{
    int k = (x % N + N) % N;
    for (int i = h[k]; i != -1; i = ne[i])
        if (e[i] == x)
            return true;

    return false;
}
\end{lstlisting}
\begin{lstlisting}
//开放寻址法
int h[N];

// 如果x在哈希表中,返回x的下标;如果x不在哈希表中,返回x应该插入的位置
int find(int x)
{
    int t = (x % N + N) % N;
    while (h[t] != null && h[t] != x)
    {
        t ++ ;
        if (t == N) t = 0;
    }
    return t;
}
\end{lstlisting}
\subsection{字符串哈希}
\begin{lstlisting}
//核心思想:将字符串看成P进制数,P的经验值是131或13331,取这两个值的冲突概率低
//小技巧:取模的数用2^64,这样直接用unsigned long long存储,溢出的结果就是取模的结果

typedef unsigned long long ULL;
ULL h[N], p[N]; // h[k]存储字符串前k个字母的哈希值, p[k]存储 P^k mod 2^64

// 初始化
p[0] = 1;
for (int i = 1; i <= n; i ++ )
{
    h[i] = h[i - 1] * P + str[i];
    p[i] = p[i - 1] * P;
}

// 计算子串 str[l ~ r] 的哈希值
ULL get(int l, int r)
{
    return h[r] - h[l - 1] * p[r - l + 1];
}

\end{lstlisting}
\section{C++STL}
\begin{lstlisting}
vector, 变长数组,倍增的思想
    size()  返回元素个数
    empty()  返回是否为空
    clear()  清空
    front()/back()
    push_back()/pop_back()
    begin()/end()
    []
    支持比较运算,按字典序

pair<int, int>
    first, 第一个元素
    second, 第二个元素
    支持比较运算,以first为第一关键字,以second为第二关键字(字典序)

string,字符串
    size()/length()  返回字符串长度
    empty()
    clear()
    substr(起始下标,(子串长度))  返回子串
    c_str()  返回字符串所在字符数组的起始地址

queue, 队列
    size()
    empty()
    push()  向队尾插入一个元素
    front()  返回队头元素
    back()  返回队尾元素
    pop()  弹出队头元素

priority_queue, 优先队列,默认是大根堆
    size()
    empty()
    push()  插入一个元素
    top()  返回堆顶元素
    pop()  弹出堆顶元素
    定义成小根堆的方式:priority_queue<int, vector<int>, greater<int>> q;

stack, 栈
    size()
    empty()
    push()  向栈顶插入一个元素
    top()  返回栈顶元素
    pop()  弹出栈顶元素

deque, 双端队列
    size()
    empty()
    clear()
    front()/back()
    push_back()/pop_back()
    push_front()/pop_front()
    begin()/end()
    []

set, map, multiset, multimap, 基于平衡二叉树(红黑树),动态维护有序序列
    size()
    empty()
    clear()
    begin()/end()
    ++, -- 返回前驱和后继,时间复杂度 O(logn)

    set/multiset
        insert()  插入一个数
        find()  查找一个数
        count()  返回某一个数的个数
        erase()
            (1) 输入是一个数x,删除所有x   O(k + logn)
            (2) 输入一个迭代器,删除这个迭代器
        lower_bound()/upper_bound()
            lower_bound(x)  返回大于等于x的最小的数的迭代器
            upper_bound(x)  返回大于x的最小的数的迭代器
    map/multimap
        insert()  插入的数是一个pair
        erase()  输入的参数是pair或者迭代器
        find()
        []  注意multimap不支持此操作。 时间复杂度是 O(logn)
        lower_bound()/upper_bound()

unordered_set, unordered_map, unordered_multiset, unordered_multimap, 哈希表
    和上面类似,增删改查的时间复杂度是 O(1)
    不支持 lower_bound()/upper_bound(), 迭代器的++,--

bitset, 圧位
    bitset<10000> s;
    ~, &, |, ^
    >>, <<
    ==, !=
    []

    count()  返回有多少个1

    any()  判断是否至少有一个1
    none()  判断是否全为0

    set()  把所有位置成1
    set(k, v)  将第k位变成v
    reset()  把所有位变成0
    flip()  等价于~
    flip(k) 把第k位取反
\end{lstlisting}

\chapter{搜索与图论}
\section{存图}
\subsection{邻接矩阵}
\begin{lstlisting}
//二维矩阵储存点之间的连接
g[a][b] = w;
\end{lstlisting}
\subsection{链表向前星}
\begin{lstlisting}
// 对于每个点k,开一个单链表,存储k所有可以走到的点。h[k]存储这个单链表的头结点
int h[N], e[N], ne[N], idx;

// 添加一条边a->b
void add(int a, int b)
{
    e[idx] = b, ne[idx] = h[a], h[a] = idx ++ ;
}

// 初始化
idx = 0;
memset(h, -1, sizeof h);
\end{lstlisting}
\newpage
\section{朴素搜索}
\subsection{DFS}
\begin{lstlisting}
int dfs(int u)
{
    st[u] = true; // st[u] 表示点u已经被遍历过

    for (int i = h[u]; i != -1; i = ne[i])
    {
        int j = e[i];
        if (!st[j]) dfs(j);
    }
}
\end{lstlisting}
\subsection{BFS}
\begin{lstlisting}
queue<int> q;
st[1] = true; // 表示1号点已经被遍历过
q.push(1);

while (q.size())
{
    int t = q.front();
    q.pop();

    for (int i = h[t]; i != -1; i = ne[i])
    {
        int j = e[i];
        if (!st[j])
        {
            st[j] = true; // 表示点j已经被遍历过
            q.push(j);
        }
    }
}
\end{lstlisting}
\subsection{拓扑排序}
\begin{lstlisting}
bool topsort()
{
    int hh = 0, tt = -1;

    // d[i] 存储点i的入度
    for (int i = 1; i <= n; i ++ )
        if (!d[i])
            q[ ++ tt] = i;

    while (hh <= tt)
    {
        int t = q[hh ++ ];

        for (int i = h[t]; i != -1; i = ne[i])
        {
            int j = e[i];
            if (-- d[j] == 0)
                q[ ++ tt] = j;
        }
    }

    // 如果所有点都入队了,说明存在拓扑序列;否则不存在拓扑序列。
    return tt == n - 1;
}
\end{lstlisting}

\section{单源最短路}
\subsection{Dijkstra}
\begin{lstlisting}
//朴素Dijkstra
int g[N][N];  // 存储每条边
int dist[N];  // 存储1号点到每个点的最短距离
bool st[N];   // 存储每个点的最短路是否已经确定

// 求1号点到n号点的最短路,如果不存在则返回-1
int dijkstra()
{
    memset(dist, 0x3f, sizeof dist);
    dist[1] = 0;

    for (int i = 0; i < n - 1; i ++ )
    {
        int t = -1;     // 在还未确定最短路的点中,寻找距离最小的点
        for (int j = 1; j <= n; j ++ )
            if (!st[j] && (t == -1 || dist[t] > dist[j]))
                t = j;

        // 用t更新其他点的距离
        for (int j = 1; j <= n; j ++ )
            dist[j] = min(dist[j], dist[t] + g[t][j]);

        st[t] = true;
    }

    if (dist[n] == 0x3f3f3f3f) return -1;
    return dist[n];
}
\end{lstlisting}
\begin{lstlisting}
//堆优化版本Dijkstra
typedef pair<int, int> PII;

int n;      // 点的数量
int h[N], w[N], e[N], ne[N], idx;       // 邻接表存储所有边
int dist[N];        // 存储所有点到1号点的距离
bool st[N];     // 存储每个点的最短距离是否已确定

// 求1号点到n号点的最短距离,如果不存在,则返回-1
int dijkstra()
{
    memset(dist, 0x3f, sizeof dist);
    dist[1] = 0;
    priority_queue<PII, vector<PII>, greater<PII>> heap;
    heap.push({0, 1});      // first存储距离,second存储节点编号

    while (heap.size())
    {
        auto t = heap.top();
        heap.pop();

        int ver = t.second, distance = t.first;

        if (st[ver]) continue;
        st[ver] = true;

        for (int i = h[ver]; i != -1; i = ne[i])
        {
            int j = e[i];
            if (dist[j] > distance + w[i])
            {
                dist[j] = distance + w[i];
                heap.push({dist[j], j});
            }
        }
    }

    if (dist[n] == 0x3f3f3f3f) return -1;
    return dist[n];
}
\end{lstlisting}
\subsection{Bellman-Ford}
\begin{lstlisting}
int n, m;       // n表示点数,m表示边数
int dist[N];        // dist[x]存储1到x的最短路距离

struct Edge     // 边,a表示出点,b表示入点,w表示边的权重
{
    int a, b, w;
}edges[M];

// 求1到n的最短路距离,如果无法从1走到n,则返回-1。
int bellman_ford()
{
    memset(dist, 0x3f, sizeof dist);
    dist[1] = 0;

    // 如果第n次迭代仍然会松弛三角不等式,就说明存在一条长度是n+1的最短路径,由抽屉原理,路径中至少存在两个相同的点,说明图中存在负权回路。
    for (int i = 0; i < n; i ++ )
    {
        for (int j = 0; j < m; j ++ )
        {
            int a = edges[j].a, b = edges[j].b, w = edges[j].w;
            if (dist[b] > dist[a] + w)
                dist[b] = dist[a] + w;
        }
    }

    if (dist[n] > 0x3f3f3f3f / 2) return -1;
    return dist[n];
}
\end{lstlisting}
\subsection{SPFA(尽可能避免SPFA!)}
\begin{lstlisting}
//SPFA查找最短路
int n;      // 总点数
int h[N], w[N], e[N], ne[N], idx;       // 邻接表存储所有边
int dist[N];        // 存储每个点到1号点的最短距离
bool st[N];     // 存储每个点是否在队列中

// 求1号点到n号点的最短路距离,如果从1号点无法走到n号点则返回-1
int spfa()
{
    memset(dist, 0x3f, sizeof dist);
    dist[1] = 0;

    queue<int> q;
    q.push(1);
    st[1] = true;

    while (q.size())
    {
        auto t = q.front();
        q.pop();

        st[t] = false;

        for (int i = h[t]; i != -1; i = ne[i])
        {
            int j = e[i];
            if (dist[j] > dist[t] + w[i])
            {
                dist[j] = dist[t] + w[i];
                if (!st[j])     // 如果队列中已存在j,则不需要将j重复插入
                {
                    q.push(j);
                    st[j] = true;
                }
            }
        }
    }

    if (dist[n] == 0x3f3f3f3f) return -1;
    return dist[n];
}
\end{lstlisting}
\begin{lstlisting}
//SPFA判定负权环
int n;      // 总点数
int h[N], w[N], e[N], ne[N], idx;       // 邻接表存储所有边
int dist[N], cnt[N];        // dist[x]存储1号点到x的最短距离,cnt[x]存储1到x的最短路中经过的点数
bool st[N];     // 存储每个点是否在队列中

// 如果存在负环,则返回true,否则返回false。
bool spfa()
{
    // 不需要初始化dist数组
    // 原理:如果某条最短路径上有n个点(除了自己),那么加上自己之后一共有n+1个点,由抽屉原理一定有两个点相同,所以存在环。

    queue<int> q;
    for (int i = 1; i <= n; i ++ )
    {
        q.push(i);
        st[i] = true;
    }

    while (q.size())
    {
        auto t = q.front();
        q.pop();

        st[t] = false;

        for (int i = h[t]; i != -1; i = ne[i])
        {
            int j = e[i];
            if (dist[j] > dist[t] + w[i])
            {
                dist[j] = dist[t] + w[i];
                cnt[j] = cnt[t] + 1;
                if (cnt[j] >= n) return true;       // 如果从1号点到x的最短路中包含至少n个点(不包括自己),则说明存在环
                if (!st[j])
                {
                    q.push(j);
                    st[j] = true;
                }
            }
        }
    }

    return false;
}
\end{lstlisting}
\section{全源最短路}
\subsection{Floyd算法}
\begin{lstlisting}
//初始化:
for (int i = 1; i <= n; i ++ )
    for (int j = 1; j <= n; j ++ )
        if (i == j) d[i][j] = 0;
        else d[i][j] = INF;

// 算法结束后,d[a][b]表示a到b的最短距离
void floyd()
{
for (int k = 1; k <= n; k ++ )
    for (int i = 1; i <= n; i ++ )
        for (int j = 1; j <= n; j ++ )
            d[i][j] = min(d[i][j], d[i][k] + d[k][j]);
}
\end{lstlisting}

\section{最小生成树}
\subsection{Prim算法}
\begin{lstlisting}
int n;      // n表示点数
int g[N][N];        // 邻接矩阵,存储所有边
int dist[N];        // 存储其他点到当前最小生成树的距离
bool st[N];     // 存储每个点是否已经在生成树中


// 如果图不连通,则返回INF(值是0x3f3f3f3f), 否则返回最小生成树的树边权重之和
int prim()
{
    memset(dist, 0x3f, sizeof dist);

    int res = 0;
    for (int i = 0; i < n; i ++ )
    {
        int t = -1;
        for (int j = 1; j <= n; j ++ )
            if (!st[j] && (t == -1 || dist[t] > dist[j]))
                t = j;

        if (i && dist[t] == INF) return INF;

        if (i) res += dist[t];
        st[t] = true;

        for (int j = 1; j <= n; j ++ ) dist[j] = min(dist[j], g[t][j]);
    }

    return res;
}
\end{lstlisting}
\subsection{Kruskal算法}
\begin{lstlisting}
int n, m;       // n是点数,m是边数
int p[N];       // 并查集的父节点数组

struct Edge     // 存储边
{
    int a, b, w;

    bool operator< (const Edge &W)const
    {
        return w < W.w;
    }
}edges[M];

int find(int x)     // 并查集核心操作
{
    if (p[x] != x) p[x] = find(p[x]);
    return p[x];
}

int kruskal()
{
    sort(edges, edges + m);

    for (int i = 1; i <= n; i ++ ) p[i] = i;    // 初始化并查集

    int res = 0, cnt = 0;
    for (int i = 0; i < m; i ++ )
    {
        int a = edges[i].a, b = edges[i].b, w = edges[i].w;

        a = find(a), b = find(b);
        if (a != b)     // 如果两个连通块不连通,则将这两个连通块合并
        {
            p[a] = b;
            res += w;
            cnt ++ ;
        }
    }

    if (cnt < n - 1) return INF;
    return res;
}
\end{lstlisting}

\section{二分图}
\subsection{染色法判定二分图:不存在奇环}
\begin{lstlisting}
int n;      // n表示点数
int h[N], e[M], ne[M], idx;     // 邻接表存储图
int color[N];       // 表示每个点的颜色,-1表示未染色,0表示白色,1表示黑色

// 参数:u表示当前节点,c表示当前点的颜色
bool dfs(int u, int c)
{
    color[u] = c;
    for (int i = h[u]; i != -1; i = ne[i])
    {
        int j = e[i];
        if (color[j] == -1)
        {
            if (!dfs(j, !c)) return false;
        }
        else if (color[j] == c) return false;
    }

    return true;
}

bool check()
{
    memset(color, -1, sizeof color);
    bool flag = true;
    for (int i = 1; i <= n; i ++ )
        if (color[i] == -1)
            if (!dfs(i, 0))
            {
                flag = false;
                break;
            }
    return flag;
}
\end{lstlisting}

\subsection{匈牙利算法:二分图最小匹配}
\begin{lstlisting}
int n1, n2;     // n1表示第一个集合中的点数,n2表示第二个集合中的点数
int h[N], e[M], ne[M], idx;     // 邻接表存储所有边,匈牙利算法中只会用到从第一个集合指向第二个集合的边,所以这里只用存一个方向的边
int match[N];       // 存储第二个集合中的每个点当前匹配的第一个集合中的点是哪个
bool st[N];     // 表示第二个集合中的每个点是否已经被遍历过

bool find(int x)
{
    for (int i = h[x]; i != -1; i = ne[i])
    {
        int j = e[i];
        if (!st[j])
        {
            st[j] = true;
            if (match[j] == 0 || find(match[j]))
            {
                match[j] = x;
                return true;
            }
        }
    }

    return false;
}

// 求最大匹配数,依次枚举第一个集合中的每个点能否匹配第二个集合中的点
int res = 0;
for (int i = 1; i <= n1; i ++ )
{
    memset(st, false, sizeof st);
    if (find(i)) res ++ ;
}
\end{lstlisting}

\chapter{数学}
\section{质数判定}
\subsection{试除法}
\begin{lstlisting}
bool is_prime(int x)
{
    if (x < 2) return false;
    for (int i = 2; i <= x / i; i ++ )
        if (x % i == 0)
            return false;
    return true;
}
\end{lstlisting}

\section{分解因数}
\subsection{试除法}
\begin{lstlisting}
void divide(int x)
{
    for (int i = 2; i <= x / i; i ++ )
        if (x % i == 0)
        {
            int s = 0;
            while (x % i == 0) x /= i, s ++ ;
            cout << i << ' ' << s << endl;
        }
    if (x > 1) cout << x << ' ' << 1 << endl;
    cout << endl;
}
\end{lstlisting}

\section{筛法}
\subsection{朴素筛法求素数}
\begin{lstlisting}
int primes[N], cnt;     // primes[]存储所有素数
bool st[N];         // st[x]存储x是否被筛掉

void get_primes(int n)
{
    for (int i = 2; i <= n; i ++ )
    {
        if (st[i]) continue;
        primes[cnt ++ ] = i;
        for (int j = i + i; j <= n; j += i)
            st[j] = true;
    }
}
\end{lstlisting}
\subsection{线性筛法求素数}
\begin{lstlisting}
int primes[N], cnt;     // primes[]存储所有素数
bool st[N];         // st[x]存储x是否被筛掉

void get_primes(int n)
{
    for (int i = 2; i <= n; i ++ )
    {
        if (!st[i]) primes[cnt ++ ] = i;
        for (int j = 0; primes[j] <= n / i; j ++ )
        {
            st[primes[j] * i] = true;
            if (i % primes[j] == 0) break;
        }
    }
}
\end{lstlisting}

\section{约数}
\subsection{试除法求约数}
\begin{lstlisting}
vector<int> get_divisors(int x)
{
    vector<int> res;
    for (int i = 1; i <= x / i; i ++ )
        if (x % i == 0)
        {
            res.push_back(i);
            if (i != x / i) res.push_back(x / i);
        }
    sort(res.begin(), res.end());
    return res;
}
\end{lstlisting}
\subsection{约数个数和约数之和}
\begin{lstlisting}
//如果 N = p1^c1 * p2^c2 * ... *pk^ck
//约数个数
(c1 + 1) * (c2 + 1) * ... * (ck + 1)
//约数之和
(p1^0 + p1^1 + ... + p1^c1) * ... * (pk^0 + pk^1 + ... + pk^ck)
\end{lstlisting}

\section{欧几里得算法:求最大公约数}
\subsection{普通欧几里得算法}
\begin{lstlisting}
int gcd(int a, int b)
{
    return b ? gcd(b, a % b) : a;
}
\end{lstlisting}
\subsection{扩展欧几里得算法}
\begin{lstlisting}
// 求x, y,使得ax + by = gcd(a, b)
int exgcd(int a, int b, int &x, int &y)
{
    if (!b)
    {
        x = 1; y = 0;
        return a;
    }
    int d = exgcd(b, a % b, y, x);
    y -= (a/b) * x;
    return d;
}
\end{lstlisting}

\section{欧拉函数}
\subsection{朴素求欧拉函数}
\begin{lstlisting}
int phi(int x)
{
    int res = x;
    for (int i = 2; i <= x / i; i ++ )
        if (x % i == 0)
        {
            res = res / i * (i - 1);
            while (x % i == 0) x /= i;
        }
    if (x > 1) res = res / x * (x - 1);

    return res;
}
\end{lstlisting}
\subsection{筛法求欧拉函数}
\begin{lstlisting}
int primes[N], cnt;     // primes[]存储所有素数
int euler[N];           // 存储每个数的欧拉函数
bool st[N];         // st[x]存储x是否被筛掉


void get_eulers(int n)
{
    euler[1] = 1;
    for (int i = 2; i <= n; i ++ )
    {
        if (!st[i])
        {
            primes[cnt ++ ] = i;
            euler[i] = i - 1;
        }
        for (int j = 0; primes[j] <= n / i; j ++ )
        {
            int t = primes[j] * i;
            st[t] = true;
            if (i % primes[j] == 0)
            {
                euler[t] = euler[i] * primes[j];
                break;
            }
            euler[t] = euler[i] * (primes[j] - 1);
        }
    }
}
\end{lstlisting}

\section{高斯消元法}
\begin{lstlisting}
// a[N][N]是增广矩阵
int gauss()
{
    int c, r;
    for (c = 0, r = 0; c < n; c ++ )
    {
        int t = r;
        for (int i = r; i < n; i ++ )   // 找到绝对值最大的行
            if (fabs(a[i][c]) > fabs(a[t][c]))
                t = i;

        if (fabs(a[t][c]) < eps) continue;

        for (int i = c; i <= n; i ++ ) swap(a[t][i], a[r][i]);      // 将绝对值最大的行换到最顶端
        for (int i = n; i >= c; i -- ) a[r][i] /= a[r][c];      // 将当前行的首位变成1
        for (int i = r + 1; i < n; i ++ )       // 用当前行将下面所有的列消成0
            if (fabs(a[i][c]) > eps)
                for (int j = n; j >= c; j -- )
                    a[i][j] -= a[r][j] * a[i][c];

        r ++ ;
    }

    if (r < n)
    {
        for (int i = r; i < n; i ++ )
            if (fabs(a[i][n]) > eps)
                return 2; // 无解
        return 1; // 有无穷多组解
    }

    for (int i = n - 1; i >= 0; i -- )
        for (int j = i + 1; j < n; j ++ )
            a[i][n] -= a[i][j] * a[j][n];

    return 0; // 有唯一解
}
\end{lstlisting}

\section{求组合数}
\subsection{递推求组合数}
\begin{lstlisting}
// c[a][b] 表示从a个苹果中选b个的方案数
for (int i = 0; i < N; i ++ )
    for (int j = 0; j <= i; j ++ )
        if (!j) c[i][j] = 1;
        else c[i][j] = (c[i - 1][j] + c[i - 1][j - 1]) % mod;
\end{lstlisting}
\subsection{预处理逆元求组合数}
\begin{lstlisting}
//首先预处理出所有阶乘取模的余数fact[N],以及所有阶乘取模的逆元infact[N]
//如果取模的数是质数,可以用费马小定理求逆元
int qmi(int a, int k, int p)    // 快速幂模板
{
    int res = 1;
    while (k)
    {
        if (k & 1) res = (LL)res * a % p;
        a = (LL)a * a % p;
        k >>= 1;
    }
    return res;
}

// 预处理阶乘的余数和阶乘逆元的余数
fact[0] = infact[0] = 1;
for (int i = 1; i < N; i ++ )
{
    fact[i] = (LL)fact[i - 1] * i % mod;
    infact[i] = (LL)infact[i - 1] * qmi(i, mod - 2, mod) % mod;
}
\end{lstlisting}

\subsection{Lucas定理求组合数}
\begin{lstlisting}
//若p是质数,则对于任意整数 1 <= m <= n,有:
//C(n, m) = C(n % p, m % p) * C(n / p, m / p) (mod p)

int qmi(int a, int k, int p)  // 快速幂模板
{
    int res = 1 % p;
    while (k)
    {
        if (k & 1) res = (LL)res * a % p;
        a = (LL)a * a % p;
        k >>= 1;
    }
    return res;
}

int C(int a, int b, int p)  // 通过定理求组合数C(a, b)
{
    if (a < b) return 0;

    LL x = 1, y = 1;  // x是分子,y是分母
    for (int i = a, j = 1; j <= b; i --, j ++ )
    {
        x = (LL)x * i % p;
        y = (LL) y * j % p;
    }

    return x * (LL)qmi(y, p - 2, p) % p;
}

int lucas(LL a, LL b, int p)
{
    if (a < p && b < p) return C(a, b, p);
    return (LL)C(a % p, b % p, p) * lucas(a / p, b / p, p) % p;
}
\end{lstlisting}

\subsection{分解质因数求组合数}
\begin{lstlisting}
//当我们需要求出组合数的真实值,而非对某个数的余数时,分解质因数的方式比较好用:
//1. 筛法求出范围内的所有质数
//2. 通过 C(a, b) = a! / b! / (a - b)! 这个公式求出每个质因子的次数。 n! 中p的次数是 n / p + n / p^2 + n / p^3 + ...
//3. 用高精度乘法将所有质因子相乘

int primes[N], cnt;     // 存储所有质数
int sum[N];     // 存储每个质数的次数
bool st[N];     // 存储每个数是否已被筛掉


void get_primes(int n)      // 线性筛法求素数
{
    for (int i = 2; i <= n; i ++ )
    {
        if (!st[i]) primes[cnt ++ ] = i;
        for (int j = 0; primes[j] <= n / i; j ++ )
        {
            st[primes[j] * i] = true;
            if (i % primes[j] == 0) break;
        }
    }
}


int get(int n, int p)       // 求n!中的次数
{
    int res = 0;
    while (n)
    {
        res += n / p;
        n /= p;
    }
    return res;
}


vector<int> mul(vector<int> a, int b)       // 高精度乘低精度模板
{
    vector<int> c;
    int t = 0;
    for (int i = 0; i < a.size(); i ++ )
    {
        t += a[i] * b;
        c.push_back(t % 10);
        t /= 10;
    }

    while (t)
    {
        c.push_back(t % 10);
        t /= 10;
    }

    return c;
}

get_primes(a);  // 预处理范围内的所有质数

for (int i = 0; i < cnt; i ++ )     // 求每个质因数的次数
{
    int p = primes[i];
    sum[i] = get(a, p) - get(b, p) - get(a - b, p);
}

vector<int> res;
res.push_back(1);

for (int i = 0; i < cnt; i ++ )     // 用高精度乘法将所有质因子相乘
    for (int j = 0; j < sum[i]; j ++ )
        res = mul(res, primes[i]);
\end{lstlisting}

\subsection{卡特兰数}
\begin{lstlisting}
//给定n个0和n个1,它们按照某种顺序排成长度为2n的序列,满足任意前缀中0的个数都不少于1的个数的序列的数量为
Cat(n) = C(2n, n) / (n + 1);
\end{lstlisting}

\section{公平组合博弈(ICG)}
\subsection{NIM博弈}
给定$N$堆物品,第i堆物品有$A_i$个。两名玩家轮流行动,每次可以任选一堆,取走任意多个物品,可把一堆取光,但不能不取。取走最后一件物品者获胜。两人都采取最优策略,问先手是否必胜。
我们把这种游戏称为NIM博弈。把游戏过程中面临的状态称为局面。整局游戏第一个行动的称为先手,第二个行动的称为后手。若在某一局面下无论采取何种行动,都会输掉游戏,则称该局面必败。
所谓采取最优策略是指,若在某一局面下存在某种行动,使得行动后对面面临必败局面,则优先采取该行动。同时,这样的局面被称为必胜。我们讨论的博弈问题一般都只考虑理想情况,即两人均无失误,都采取最优策略行动时游戏的结果。
NIM博弈不存在平局,只有先手必胜和先手必败两种情况。
\begin{theorem}
NIM博弈先手必胜,当且仅当 $A_1 \oplus A_2 \oplus … \oplus A_n != 0$
\end{theorem}
\subsection{公平组合博弈}
若一个游戏满足:
\begin{itemize}
\item 由两名玩家交替行动;
\item 在游戏进程的任意时刻,可以执行的合法行动与轮到哪名玩家无关;
\item 不能行动的玩家判负;
\end{itemize}
则称该游戏为一个公平组合游戏。NIM博弈属于公平组合游戏,但城建的棋类游戏,比如围棋,就不是公平组合游戏。
因为围棋交战双方分别只能落黑子和白子,胜负判定也比较复杂,不满足条件2和条件3。

\end{document}